%\documentclass[11pt, onecolumn, preprint, nocopyrightspace]{sigplanconf}
\documentclass[10pt, preprint, nocopyrightspace]{sigplanconf}

\title{The Universal \oh Machine\\\large Recursion Theory\\ Homework 4}

\authorinfo{Spenser Bauman \and Tori Lewis \and Cameron Swords}
           {Indiana University}
           {http://github.com/sabauma/OneHash}

\usepackage{amsmath, listings, amsthm, amssymb, proof, mathabx}
\usepackage{textcomp} \usepackage{seqsplit} \usepackage{tikz}
\usepackage{fix2col} \usepackage{stmaryrd} \usepackage{url}
\usepackage{scrextend}
\usepackage{xspace}

\newtheorem{name}{Printed output}
\newtheorem{defn}{Definition} \newtheorem{ex}{Example} \newtheorem{ans}{Answer}
\newtheorem*{thm}{Theorem}

\newcommand{\cam}[1]{\emph{Cam: #1}}

\newenvironment{proofcases}[1][] {\textbf{\emph{Proof.}
#1}\newcommand{\case}[1]{\underline{Case [##1]:}} \begin{addmargin}[1em]{0em}
\setlength{\parskip}{1em}}
{\end{addmargin}\setlength{\parskip}{0em}\begin{flushright}$\qed$\end{flushright}}

\newcommand{\oh}{$1\#$\xspace}
\renewcommand{\phi}{\varphi}
\newcommand{\appref}[1]{Appendix \ref{#1}}
\newcommand{\figref}[1]{Figure \ref{#1}}
%\lstset{basicstyle=\sffamily\footnotesize}
\lstset{
  language=Haskell,
  basicstyle=\sffamily\footnotesize,
  breaklines=true,
  literate=
  {\\}{{$\lambda$}}1
  {<-}{{$\leftarrow$}}1
  {->}{{$\rightarrow$}}1
  {::}{{$:\!\,:$}}1
  {>>}{{$>\!>$}}1
}

\begin{document}

\maketitle

\section{Introduction and Overview}

This is an explanation and outline of an implementation of a universal program
in \oh. This work focused on semantically-clean solutions to encoding the universal
program instead of a concise, compact solution, and the result is a very small
Haskell implementation that constructs a (quite large) \oh implementation.

This universal program was developed using an embedding of \oh in Haskell
developed by Spencer Bauman earlier in the course, which has allowed us many of the
traditional comforts of Haskell in our implementation, including: notation, type
declarations, monadic-style programming, and a rich abstraction layer over the
target language. Furthermore, the ideas are clear and concise in the final
implementation.

Our general approach diverges from the canonical approach by fully encoding the input
program before usage, an implementation detail that reduces the instruction
lookup mechanism to the register lookup problem. The main loop of the program 
looks up a single instruction, runs it with the appropriate register updates,
and then loops so long as it finds an instruction at its new program counter.

In the following sections we’ll explain how each of the components of this
program work, why they correctly perform their task, and provide the Haskell
implementation along-side the discussion. When necessary, we will also present
examples. 
We use the canonical registers as proposed by Moss~\cite{moss}: $R1$ contains the
program,
$R2$ the program counter, $R3$ the current instruction, and
$R4$ the encoded registers. We also reserve $R5$ for the ``working register'',
the number of the register we're currently working on, and $R6$ for its decoded
contents.



The full \oh code and the Haskell code for embedding \oh are included in the
appendix, and the entire Haskell implementation (with resultant universal
program) can be found at 
\begin{center}
\lstinline{http://github.com/sabauma/OneHash}
\end{center}

\section{Haskell Implementation}


Rather than go through the tedious and error prone process of writing \oh code,
we made use of a Haskell DSL (which we also developed) for composing \oh
programs.
The DSL contains many useful primitives for working with \oh registers at a high
level of abstraction.
\oh code is written in Haskell as a sequence of monadic actions which produce
\oh register operations, jumps, and labels.
Jump operations may only target labels.
Once the program is fully expanded to this intermediate language, all jump
operations are converted into forward and backwards \oh jump operations, and
labels are eliminated.

The primitive interface described above makes it possible to write high-level
combinators which encode complex control flow structures in \oh (e.g. a case
operation whose branches may contain an arbitrary number of instructions with
intentional non-fall-through behaviour).
Internally, these operations expand complex control flow into a linear sequence
of labels, jumps, and \oh level case instructions.

For example, consider the code for generating a \texttt{case} operation which
reads from the given register and dispatches to the appropriate action based
on the register's contents.
\begin{lstlisting}
cases :: Reg
      -> OneHash () -- Empty register
      -> OneHash () -- Got a 1
      -> OneHash () -- Got a #
      -> OneHash ()
cases r c1 c2 c3 =
    void $ mfix $ \ ~(j1, j2, jend) -> do
    tell [Case r] >> j1 >> j2 >> c3 >> jend
    liftA3 (,,) (label <* c1 <* jend)
                (label <* c2)
                label
\end{lstlisting}
Such primitives are used to generate more complex looping structures and high level
operations like \texttt{clear}, \texttt{copy}, and \texttt{reverseReg}.

\begin{lstlisting}
lookupReg' :: Reg -> Reg -> Reg -> OneHash ()
lookupReg' rin n rout = withRegs $
  \ rin' n' -> do
    -- Copy registers to preserve their contents
    copy rin  rin'
    copy n    n'
    -- Convert to 0-indexing
    chomp n'
    -- Consume the first n things
    loop' n' (decode rin' rout) noop
    clear rout
    -- Decode the current value into rout
    decode rin' rout
\end{lstlisting}

To further explain this encoding, consider the example encoding where the
encoded sequence is $11\#\#1\#\#\#11\#\#$ and we are looking for the second
register, presented in \figref{fig:lookup}. Once again, \lstinline{withRegs}
clears out the allocated registers after usage (so, in this case,
\lstinline{rin'}) when the clause is done.

\begin{figure*}
\begin{tabular}{llp{.64\textwidth}}
$R_{in}$ & $R_{out}$ & Description\\\hline
11\#\#\#\#\#1\# &  & 
Encode does a case on the register containing $p$.
Since the first thing it eats is a 1, it writes two 1’s to out register and does
a second case on the register  P was in.\\

\\  1\#\#\#\#\#1\# &
 11 &
Again,  since it eats another 1, it writes two 1s but this time it loops back to
the start of this nested case. 
\\ \#\#\#\#\#1\# &
 1111 &
This time it eats a \#, so it writes a 1 followed by a \#. Then it does a second
nested case on the same register P was in . 
\\ \#\#\#\#1\# &
 11111\# &
Again it eats a \#, so it writes a 1 followed by a \#, and loops back to the
start of this nested case statement. The step repeats until it eats all of the
consecutive hashes. 
\\ \#\#\#1\# &
11111\#1\# &
\\1\# &
  11111\#1\#1\#1\#1\# &
When it eats a one after hashes, it writes two \#s followed by two 1s and does
the first nested case statement (labeled one loop)
\\\# &
 11111\#1\#1\#1\#1\#\#\#11 &
\\  &
 11111\#1\#1\#1\#1\#\#\#11 &
Since RIN is empty encode writes two \#s then ends.  
\\ &
  11111\#1\#1\#1\#1\#\#\#11\#\# &
\\
\end{tabular}
\caption{\label{fig:encode-prog}An encoding of the program \#\#\#\#\#1\#.}
\end{figure*}



\begin{figure*}
\begin{tabular}{lllllp{.44\textwidth}}
$R_{in}$ & $n$ & $R_{in'}$ & $n'$ & $R_{out}$ & Description\\\hline
11\#\#1\#\#\#11\#\#  & 11  & & & &
First Lookup copies the register containing the encoded sequence and the
register with the index value to two new registers.   \\

11\#\#1\#\#\#11\#\#  &
11  &
11\#\#1\#\#\#11\#\#  &
11  &
  &
The first thing that look up does it remove the first one from the new indexing
register r2, so that the last word it destroys is the (n-1)th word.    \\
11\#\#1\#\#\#11\#\#  &
1  &
11\#\#1\#\#\#11\#\#  &
1 &
  &
Then Lookup loops through (as a case on r2) decoding the words to $rout$.  \\
11\#\#1\#\#\#11\#\#  &
11  &
1\#\#\#11\#\#  &
  &
1 & 
At the end of the loop, it clears out $rout$.\\
11\#\#1\#\#\#11\#\#  &
11  &
1\#\#\#11\#\#  &
&
&
Finally Lookup decodes the nth instruction into r out  \\
11\#\#1\#\#\#11\#\#  &
11  &
11\#\#  &
&
\#  \\
\end{tabular}
\caption{\label{fig:lookup}A lookup through an encoded register sequence.}
\end{figure*}



\section{Encodings}

Before we discuss the details of encoding, we make one critical observation: if 
we encode the input using the same encoding strategy as the machine registers,
then we may use the same lookup procedure and indexing mechanism for instruction
lines and encoded registers. This is a serious simplification to the original
proposal for the instruction lookup procedure, because it allows us to cleanly
find the exact instruction we need.

We use the encoding proposed by Moss~\cite{moss}, wherein each $1$ is turned into a
$11$, each $\#$ into a $1\#$, and each instruction (and register) has its end
encoded as $\#\#$. That is, the encoding for programs (and registers) entails
adding a $1$ before each symbol of each word and a $\#\#$ between each word
(instruction or register). This is a straight-forward injection encoding, and
thus the program and register contents are perfectly preserved.

\paragraph{Register Encoding.}

Register encoding uses the \lstinline{toDataEncoding} procedure, outlined below,
to convert a single register into its encoded version, placing a $\#\#$ at the
end. Once the register value is thus encoded, it is placed into the correct
register (with the appropriate number of $\#\#$ delimiters) by
\lstinline{updateReg} below. 

\begin{lstlisting}
toDataEncoding :: Reg -> Reg -> OneHash()
toDataEncoding rin rout = do
  comment $ printf
              "encode data in %s into %s"
              (show rin) (show rout)
  loop' rin (add1 rout >> add1 rout)
            (add1 rout >> addh rout)
  addh rout
  addh rout
  comment "end data encoding"
\end{lstlisting}

We accomplish this encoding process with an loop over the input value via
\lstinline{loop}, which takes two sub-programs: the first is run when the input
register contains a $1$ and the second when it contains a $\#$. The loop
construct terminates when it has exhausted the input register.  We use this to
write a $11$ to the output register when the input register reads a $1$ and
$1\#$ when it reads a $\#$, repeatedly, until the input is empty.  Finally we
add $\#\#$ to the output register as the terminal.

\paragraph{Program Encoding.}

Program encoding follows directly from register encoding: we iterate over the
input program. Unlike register encodings, though, we must place a delimiter
at exactly each occurance of $\#1$ in the program (and a final one at the end).
To accomplish this, we perform the same encoding process as above, using three
case operations that jump to each other to track whether the last-seen value
is a $1$ or a $\#$ in order to add $\#\#$ when we have most recently read a $\#$
and then find a $1$.

\begin{lstlisting}
toEncoding :: Reg -> Reg -> OneHash ()
toEncoding r r' = withLabels $ \ start end -> mdo
  comment $ printf
              "encode program in %s into %s"
              (show r) (show r')
  cases r (addh r' >> addh r' >> end)
          (add1 r' >> add1 r' >> oneloop)
          (add1 r' >> addh r' >> hashloop)
  oneloop <- label
  cases r (addh r' >> addh r' >> end)
          (add1 r' >> add1 r' >> oneloop)
          (add1 r' >> addh r' >> hashloop)
  hashloop <- label
  cases r (addh r' >> addh r' >> end)
          (addh r' >> addh r' >>
           add1 r' >> add1 r' >> oneloop)
          (add1 r' >> addh r' >> hashloop)
  comment "end encode function"
\end{lstlisting}

To further explain this encoding, consider the example encoding when $p =
\#\#\#\#\#1\#$, presented in \figref{fig:encode-prog}. 

\paragraph{Decoding Entries.}

Decoding is the straight-forward inverse of encoding: when we see a $1$, we know
the next character will be a $1$ or $\#$ indicating the original value, and thus
case on the next character and produce the appropriate result in the output
register. If we see a $\#$, we kno w we are at the terminal and so we consume
the second $\#$ and complete the program. 
\begin{lstlisting}
decode :: Reg -> Reg -> OneHash ()
decode rin rout = do
  start <- label
  cases rin noop
    (cases rin noop
               (add1 rout >> start)
               (addh rout >> start))
    (cases rin noop noop noop)
\end{lstlisting}
We only ever need to decode one cell at a time, and so this implementation is
sufficient. This once again works using case statements, consuming the appropriate
values and dispatching appropriates with the labels. The \lstinline{noop} form here
is a no-op operation, indicating a ``filler'' operation since \lstinline{cases}
requires \emph{something} in each case. The simplist definition of \lstinline{noop} is
$1\#\#\#$.

\section{Lookups and Updates}

The lookup function retrives the $n^{th}$ sequence from an encoded sequence,
copying the results into the output register. It also preserves inputs, and thus
it starts by copying the encoded sequence and the index it is looking for into
new registers (allocated from a pool of unused registers with the
\lstinline{withRegs} form, which clears the newly-allocated register before and
after usage). It then finds the $n^{th}$ word of the encoded sequence and
decodes it into the appropriate output registers, cleaning up afterwards. This
is done below, in the perhaps-misnamed \lstinline{lookupReg'}. The \lstinline{copy}
form has the usual definition, \lstinline{loop} and \lstinline{decode} are as described
above, and \lstinline{chomp} simply consumes a single character from the given register.

\paragraph{Register Lookup.}

Register lookup follows directly from the above definition, specialized to work on registers
$R4$--$R6$, where $R4$ contains all registers, $R5$ contains the number of the register to
look up, and $R6$ is where the result of the lookup is stored.

\begin{lstlisting}
lookupReg :: OneHash ()
lookupReg = lookupReg' r4 r5 r6
\end{lstlisting}

\paragraph{Instruction Lookup.}

Program lookup also follows directly from the above definition, specialized to
work on registers $R1$--$R3$, where $R1$ contains the program, $R2$ contains
the instruction number, and $R3$ is where the result of the lookup is stored.

\begin{lstlisting}
lookupInstr :: OneHash ()
lookupInstr = withRegs $ \rtmp -> do
  lookupReg' r1 r2 rtmp
  reverseReg rtmp r3
\end{lstlisting}

Unlike register lookup, we must also reverse the instruction (see
\lstinline{step} below), so we do that during instruction lookup.

\paragraph{Register Update.}

First, we note that we never need to update the program: it is fixed at the
start of the run. We write a procedure that will update a single register, given a register value in $R6$, a number to update in $R5$, and the sequence of registers in $R4$.

\begin{lstlisting}
updateReg :: OneHash ()
updateReg = updateReg' r4 r5 r6
\end{lstlisting}

Like the lookup procedures, it copies everything to ``safe'' registers and loops up to the
$(n-1)^{th}$ word. Update preserves what was encoded via \lstinline{eatCell},
however, into a temporary register.
Then it deletes (or ``eats'') the the $n^{th}$ word.
Next we encode the new register value (which is decoded in $R6$) and place it
into the temporary register that contains the currently-seen cells. Finally we
move what was left of the original encoded register to the end of the temporary
register and moves the temporary to where encoded register was. 

\begin{minipage}{.48\textwidth}
\begin{lstlisting}
updateReg' :: Reg -> Reg -> Reg -> OneHash ()
updateReg' reg n val = withRegs $
  \ n' tmp -> do
    copy n n'
    chomp n'
    -- Seek to the desired register
    loop' n'
          (eatCell reg tmp >>
           addh tmp >>
           addh tmp)
          noop
    -- Consume the current register
    -- and throw it away
    withRegs $ eatCell reg
    toDataEncoding val tmp
    -- Move everything back
    move reg tmp
    move tmp reg

-- Eats a cell in the encoding, but does not
-- decode it.  This will not save the trailing
-- termination character.
eatCell :: Reg -> Reg -> OneHash ()
eatCell rin rout = do
  start <- label
  cases rin noop
    (add1 rout >>
     cases rin noop
               (add1 rout >> start)
               (addh rout >> start))
    (cases rin noop noop noop)
\end{lstlisting}
\end{minipage}

\section{Program Steps}

Since we know that instructions in \oh either manipulate a register or jump
forwards or backwards in the program, we based our design for a step function on
that. We parse the correct number of $\#$s (recall that the instruction is
reversed before \lstinline{step} is run) and use combinations of
\lstinline{lookupReg}, \lstinline{updateReg}, and \lstinline{move} and
destructive subtraction \lstinline{subDestructive} (with their usual
definitions) to do this. This procedure is provided in \figref{fig:step}. 

Step either simply adjust the line number index we are looking at or it
manipulates the content of $R6$ according to the given instruction. 
In practice this is the same as manipulating the $Rn$ referred to in the
instruction since we use lookup to find the encoded register and put it in $R6$
and update to put it back into the encoded sequence of registers. 

In the case that we must add a one or a hash to a
register, we simply look it up, add the symbol, and then update it. For jumps,
we simply adjust $R2$ by the appropriate amount, for for cases we look up the
register, use \oh's case, and adjust $R2$ accordingly to jump to the correct
line.

\begin{figure*}
\begin{lstlisting}
step :: OneHash ()
step = withLabels $ \ start end -> mdo
  comment "begin step function"
  clear r5
  cases r3 end end  noop
  cases r3 end (add1 r5 >> move r3 r5 >> write1  end) noop
  cases r3 end (add1 r5 >> move r3 r5 >> writeh  end) noop
  cases r3 end (add1 r5 >> move r3 r5 >> jumpadd end) noop
  cases r3 end (add1 r5 >> move r3 r5 >> jumpsub end) noop
  cases r3 end (add1 r5 >> move r3 r5 >> caseh   end) noop
  comment "end step function"
    where
      write1  end = lookupReg >> add1 r6 >> updateReg >> add1 r2 >> end
      writeh  end = lookupReg >> addh r6 >> updateReg >> add1 r2 >> end
      jumpadd end = move r5 r2 >> end
      jumpsub end = subDestructive r2 r5 >> end
      caseh   end = lookupReg >> cases r6 (add1 r2)
                                   (add1 r2 >> add1 r2)
                                   (add1 r2 >> add1 r2 >> add1 r2)
                              >> updateReg >> end
\end{lstlisting}
\caption{\label{fig:step}The main step relation for the universal \oh machine.}
\end{figure*}

\section{Main Loop}

The main loop of the program works in three parts: we initialize the
application, which mostly just entails encoding the program for easy lookup.
Next we begin a loop, looking up the instruction based on the program counter
in $R2$. If we find an instruction, we run it and loop. If we do not, we
are out of bounds of our instruction set and are thus done running.
Here we don't case if our program is ``tidy'': we can run any program, halting
the moment we step out of bounds. Finally we perform clean-up, placing the result
in the correct place and clearing the other registers we have used.

\paragraph{Initialization.}

As descriped, initialization entails encoding the input program in $R1$ and
replacing it back into $R1$.

\begin{lstlisting}
init :: OneHash ()
init = do
  toEncoding r1 r2
  move r2 r1
  add1 r2
\end{lstlisting}

\paragraph{Main Loop.}

As described, the main loop looks up an instruction, compares it to a new
register (guaranteed to be empty) to see if it exists, and runs the instruction
if it is. The \lstinline{cmp} funcion does precisely that: in the case the two
are equal it runs the first of the two instructions (in this case, a jump to
the label \lstinline{end}), and otherwise it runs the second, which here falls
through to the step procedure. After a step, it restarts the loop.

\begin{lstlisting}
mainLoop :: OneHash ()
mainLoop = withRegs $ \ (emptyreg :: Reg) -> mdo
  init
  loop <- label
  lookupInstr
  cmp r3 emptyreg (end) (noop)
  step
  loop
  end <- label
  cleanup
\end{lstlisting}


\paragraph{Cleanup and Final Answer.}

To clean up the computation and provide the final result, we look up the
contents of $R1$ in the simulated machine, move them into $R1$ in the real
machine, and then clear registers $R2$-$R5$ (since $R6$ must be clear after the
move operation).
Then the program ends normally at the $n+1$ instruction line with the answer in
$R1$.

\begin{lstlisting}
cleanup :: OneHash ()
cleanup = do
  clear r1
  clear r5
  add1 r5
  lookupReg
  move r6 r1
  clear r2
  clear r3
  clear r4
  clear r5
\end{lstlisting}

\section{Conclusion}

This machine was a straight-forward encoding of the algorithm described,
separated into a lookup step and an execution step. Encoding programs in the
same way as registers allowed us to reuse much of the same machinery. While the
Haskell implementation has grown large due to the usage of a domain-specific
language and compilation process, this approach is sound and reliable even for
hand-coded attempts at constructing the universal machine.

\paragraph{Division of Labor.} The division of labor is as follows:
\begin{itemize}
\item \textbf{Algorithmic Design.}
      Spenser Bauman, Tori Lewis, and Cameron Swords collectively developed the
      algorithm design and overall approach to implement the universal machine.
\item \textbf{Haskell Framework.}
      Spenser Bauman constructed a \oh compiler during previous homework
      assignments, including a small standard library of functions including
      operations like \lstinline{clear} and \lstinline{move}. This
      implementation was extended by Spenser Bauman and Cameron Swords during
      implementation of the universal machine.
\item \textbf{Running Code.}
      The Haskell implementation of the universal machine was constructed by
      Spenser Bauman (who write the encoding, decoding, and lookup mechanisms
      for registers) and Cameron Swords (the main step function and the main
      loop of the program) with design feedback from Tori Lewis.
\item \textbf{Write-Up.}
      The exposition of this report was developed by Tori Lewis, summarizing the
      results and explaining the final construction of the universal \oh
      machine.
\end{itemize}

\begin{thebibliography}{99}

\bibitem{moss}
  Lawrence S. Moss,
  \emph{\oh: a Text Register Machine Introduction to Computability Theory},
  \lstinline{http://www.indiana.edu/~iulg/trm/index.shtml},
  2015.

\end{thebibliography}

\clearpage

\appendix

\twocolumn[\section{Haskell Implementation}\lstinputlisting{U.Nocomm.1.hs}]

\clearpage

\twocolumn[\lstinputlisting{U.Nocomm.2.hs}]

\clearpage

\twocolumn[\lstinputlisting{U.Nocomm.3.hs}]

\clearpage

\section{Final Program}

\lstinputlisting{U.formatted.oh}

\end{document}
