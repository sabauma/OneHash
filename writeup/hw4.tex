%\documentclass[11pt, onecolumn, preprint, nocopyrightspace]{sigplanconf}
\documentclass[10pt, preprint, nocopyrightspace]{sigplanconf}

\title{The Universal \oh Machine\\\large Recursion Theory\\ Homework 4}

\authorinfo{Spenser Bauman \and Tori Lewis \and Cameron Swords}
           {Indiana University}
           {github.link.to.appear}

\usepackage{amsmath, listings, amsthm, amssymb, proof, mathabx}
\usepackage{textcomp} \usepackage{seqsplit} \usepackage{tikz}
\usepackage{fix2col} \usepackage{stmaryrd} \usepackage{url}
\usepackage{scrextend} 
\usepackage{xspace} 

\newtheorem{name}{Printed output}
\newtheorem{defn}{Definition} \newtheorem{ex}{Example} \newtheorem{ans}{Answer}
\newtheorem*{thm}{Theorem}

\newcommand{\cam}[1]{\emph{Cam: #1}}

\newenvironment{proofcases}[1][] {\textbf{\emph{Proof.}
#1}\newcommand{\case}[1]{\underline{Case [##1]:}} \begin{addmargin}[1em]{0em}
\setlength{\parskip}{1em}}
{\end{addmargin}\setlength{\parskip}{0em}\begin{flushright}$\qed$\end{flushright}}

\newcommand{\oh}{$1\#$\xspace} 
\renewcommand{\phi}{\varphi}
\newcommand{\appref}[1]{Appendix \ref{#1}}
%\lstset{basicstyle=\sffamily\footnotesize} 
\lstset{
  language=Haskell,
  basicstyle=\sffamily\footnotesize,
  literate=
  {\\}{{$\lambda$}}1
  {<-}{{$\leftarrow$}}1
  {->}{{$\rightarrow$}}1
  {::}{{$:\!\,:$}}1
} 

\begin{document} 

\maketitle

\section{Introduction and Overview}

\section{Encodings}

\paragraph{Register Encoding.}

\begin{lstlisting}
toEncoding :: Reg -> Reg -> OneHash ()
toEncoding r r' = withLabels $ \ start end -> mdo
  comment $ printf 
              "encode program in %s into %s" 
              (show r) (show r')
  cases r (addh r' >> addh r' >> end)
          (add1 r' >> add1 r' >> oneloop)
          (add1 r' >> addh r' >> hashloop)
  oneloop <- label
  cases r (addh r' >> addh r' >> end)
          (add1 r' >> add1 r' >> oneloop)
          (add1 r' >> addh r' >> hashloop)
  hashloop <- label
  cases r (addh r' >> addh r' >> end)
          (addh r' >> addh r' >> 
           add1 r' >> add1 r' >> oneloop)
          (add1 r' >> addh r' >> hashloop)
  comment "end encode function"

-- Generate the desired encoding of data, which 
-- is slightly different than for code, which
-- will separate instructions by ##. This 
-- converts the entire contents of rin to the 
-- encoding and places a terminator at the end.
toDataEncoding :: Reg -> Reg -> OneHash()
toDataEncoding rin rout = do
  comment $ printf 
              "encode data in %s into %s" 
              (show rin) (show rout)
  loop' rin (add1 rout >> add1 rout) 
            (add1 rout >> addh rout)
  addh rout
  addh rout
  comment "end data encoding"
\end{lstlisting}

\paragraph{Program Encoding.}

Program encoding follows directly from register encoding: ...

\paragraph{Decoding Entries.}

Any register or program instruction may be decoded as...

\begin{lstlisting}
-- Decode one cell on an encoded tape
decode :: Reg -> Reg -> OneHash ()
decode rin rout = do
  start <- label
  cases rin noop
    (cases rin noop 
               (add1 rout >> start) 
               (addh rout >> start))
    (cases rin noop noop noop)
\end{lstlisting}

\section{Lookups and Updates}

\paragraph{Register Lookup.}

Register lookup ...

\begin{lstlisting}
lookupReg :: OneHash ()
lookupReg = lookupReg' r4 r5 r6

lookupReg' :: Reg -> Reg -> Reg -> OneHash ()
lookupReg' rin n rout = withRegs $ 
  \ rin' n' -> do
    -- Copy registers to preserve their contents
    copy rin  rin'
    copy n    n'
    -- Convert to 0-indexing
    chomp n'
    -- Consume the first n things
    loop' n' (decode rin' rout) noop
    clear rout
    -- Decode the current value into rout
    decode rin' rout
\end{lstlisting}

\paragraph{Register Update.}

Register update...

\begin{lstlisting}
updateReg :: OneHash ()
updateReg = updateReg' r4 r5 r6

updateReg' :: Reg -> Reg -> Reg -> OneHash ()
updateReg' reg n val = withRegs $ 
  \ n' tmp -> do
    copy n n'
    chomp n'
    -- Seek to the desired register
    loop' n' 
          (eatCell reg tmp >> 
           addh tmp >> 
           addh tmp) 
          noop
    -- Consume the current register 
    -- and throw it away
    withRegs $ eatCell reg
    toDataEncoding val tmp
    -- Move everything back
    move reg tmp
    move tmp reg
\end{lstlisting}

\paragraph{Instruction Lookup.}

Program instruction lookup...

\begin{lstlisting}
lookupInstr :: OneHash ()
lookupInstr = withRegs $ \rtmp -> do
  lookupReg' r1 r2 rtmp
  reverseReg rtmp r3
\end{lstlisting}

\section{Program Steps}

\begin{figure*}
\begin{lstlisting}
step :: OneHash () 
step = withLabels $ \ start end -> mdo 
  comment "begin step function" 
  clear r5 
  cases r3 end end  noop 
  cases r3 end (add1 r5 >> move r3 r5 >> write1  end) noop 
  cases r3 end (add1 r5 >> move r3 r5 >> writeh  end) noop 
  cases r3 end (add1 r5 >> move r3 r5 >> jumpadd end) noop 
  cases r3 end (add1 r5 >> move r3 r5 >> jumpsub end) noop 
  cases r3 end (add1 r5 >> move r3 r5 >> caseh   end) noop 
  comment "end step function" 
    where 
      write1  end = lookupReg >> add1 r6 >> updateReg >> add1 r2 >> end 
      writeh  end = lookupReg >> addh r6 >> updateReg >> add1 r2 >> end 
      jumpadd end = move r5 r2 >> end 
      jumpsub end = subDestructive r2 r5 >> end 
      caseh   end = lookupReg >> cases r6 (add1 r2) 
                                   (add1 r2 >> add1 r2) 
                                   (add1 r2 >> add1 r2 >> add1 r2) 
                              >> updateReg >> end 
\end{lstlisting}
\caption{The main step relation for the universal \oh machine.}
\end{figure*}

\section{Main Loop}

\begin{figure}
\begin{lstlisting}
mainLoop :: OneHash ()
mainLoop = withRegs $ \ (emptyreg :: Reg) -> mdo
  init
  loop <- label
  lookupInstr
  cmp r3 emptyreg (end) (noop)
  step
  loop
  end <- label
  cleanup
\end{lstlisting}
\caption{The main loop of the universal \oh machine.}
\end{figure}

\paragraph{Initialization.}

To initialize the program, we first...

\begin{lstlisting}
init :: OneHash ()
init = do
  toEncoding r1 r2
  move r2 r1
  add1 r2
\end{lstlisting}

\paragraph{Main Loop.}

\paragraph{Cleanup and Final Answer.}

To clean up the computation and provide the final result...

\begin{lstlisting}
cleanup :: OneHash ()
cleanup = do
  clear r1
  clear r5
  add1 r5
  lookupReg
  move r6 r1
  clear r2
  clear r3
  clear r4
  clear r5
\end{lstlisting}

\section{Conclusion}

\paragraph{Division of Labor.} The division of labor is as follows:
\begin{itemize}
\item \textbf{Algorithmic Design.} 
      Spenser Bauman, Tori Lewis, and Cameron Swords collectively developed the
      algorithm design and overall approach to implement the universal machine.
\item \textbf{Haskell Framework.}
      Spenser Bauman constructed a \oh compiler during previous homework
      assignments, including a small standard library of functions including
      operations like \lstinline{clear} and \lstinline{move}. This
      implementation was extended by Spenser Bauman and Cameron Swords during
      implementation of the universal machine.
\item \textbf{Running Code.} 
      The Haskell implementation of the universal machine was constructed by
      Spenser Bauman (who write the encoding, decoding, and lookup mechanisms
      for registers) and Cameron Swords (the main step function and the main
      loop of the program) with design feedback from Tori Lewis.
\item \textbf{Write-Up.}
      The exposition of this report was developed by Tori Lewis, summarizing the
      results and explaining the final construction of the universal \oh
      machine. 
\end{itemize}

\end{document}
